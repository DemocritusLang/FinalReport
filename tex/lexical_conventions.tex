\section{Lexical Conventions}
	In this subsection, we will cover the standard lexical conventions for Democritus. As with languages such as C, Algol, or Pascal, Democritus is a free-format language. The parser will discard whitespace characters such as \texttt{` '}, \verb|\t|, and \verb|\n|.
	
	\subsection{Identifiers}
		Identifiers for Democritus will be defined as follows: any sequence of letters and numbers without whitespaces and is not a keyword will be parsed as an identifier. Note that, as in other languages, identifiers cannot begin with a number. Somewhat different, however, is the order of variable declarations; in Democritus, declarations are made following the \textit{varname vartype} structure. The regular expression defining identifiers is as follows:
		
	\begin{lstlisting}
['a'-'z' 'A'-'Z']['a'-'z' 'A'-'Z' '0'-'9' '_']*
	\end{lstlisting}
	
	\noindent An example of declarations with identifiers: 
	
	\begin{lstlisting}
2wrongID int;		/* not a valid identifier declaration */
mySecond float;		/* valid */
my_Second char;		/* valid */
	\end{lstlisting}
	
	\subsection{Literals}
		Literals, simply a sequence of numbers, may be identified with the regular expression
	\begin{lstlisting}[language=Caml]
['0'-'9']+ 					(* Int *)
['0'-'9']*'.'['0'-'9']+  	(* Float *)
	\end{lstlisting}
	
	\subsection{Tokens}
		The list of tokens used in Democritus are as follows:
		\begin{lstlisting}[language=Caml]
| '('      { LPAREN }
| ')'      { RPAREN }
| '{'      { LBRACE }
| '}'      { RBRACE }
| ';'      { SEMI }
| ':'      { COLON }
| ','      { COMMA }
| '+'      { PLUS }
| '-'      { MINUS } 
| '*'      { TIMES }
| '%'      { MOD }
| ">>"     { RSHIFT }
| "<<"     { LSHIFT }
| '/'      { DIVIDE }
| '='      { ASSIGN }
| "=="     { EQ }
| "!="     { NEQ }
| '<'      { LT }
| "<="     { LEQ }
| ">"      { GT }
| ">="     { GEQ }
| "&&"     { AND }
| "||"     { OR }
| "!"      { NOT }
| "if"     { IF }
| "else"   { ELSE }
| "elif"   { ELIF }
| "for"    { FOR }
| "return" { RETURN }
| "int"    { INT }
| "float"  { FLOAT }
| "char"   { CHAR }
| "boolean"   { BOOLEAN }
| '*' { POINTER }
| '&' { AMPERSAND }
| "function" { FUNCTION }
| "void"   { VOID }
| "struct" { STRUCT }
| "string" { STRING }
| "true"   { TRUE }
| "false"  { FALSE }
| "break"  { BREAK } 
| "continue" { CONTINUE }
| "atomic" { ATOMIC }
		\end{lstlisting}
		
		\noindent These words have been reserved by the compiler and hold special meaning within the language. Though most are self-explanatory, we will delve into their usage later on. 
		
	\subsection{Punctuation}
		\subsubsection{Semicolon}
			As in C, the semicolon `;' is required to terminate any statement in Democritus. 
			
			\begin{lstlisting}
statement SEMI
			\end{lstlisting}
			
		\subsubsection{Curly Brackets}
			In order to keep the language free-format, curly braces are used to delineate separate and nested blocks. These braces are required even for single-statement conditional and iteration loops.
			
			\begin{lstlisting}
LBRACE statements RBRACE
			\end{lstlisting}
			
		\subsubsection{Parentheses}
			To assert precedence, expressions may be encapsulated within parentheses to guarantee order of operations. 
			
			\begin{lstlisting}
LPAREN expression RPAREN
			\end{lstlisting}
			
		\subsubsection{Comments}
			For now, comments are initiated with \texttt{/*} and closed with \texttt{*/}. They cannot be nested.
