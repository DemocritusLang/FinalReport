\section{Lexical Conventions}
	In this subsection, we will cover the standard lexical conventions for Democritus. Lexical elements are scanned as `tokens,' which are then parsed into a valid Democritus program. Democritus is a free-format language, discarding all whitespace characters such as \texttt{` '}, \verb|\t|, and \verb|\n|.
	
	\subsection{Identifiers}
		Identifiers for Democritus will be defined as follows: any sequence of letters and numbers without whitespaces and not a keyword will be parsed as an identifier. Identifiers must start with a letter, but they may contain any lowercase or uppercase ASCII letter, numbers, and the underscore `\_'. Identifiers are case-sensitive, so `var1' and `Var1' would be deemed separate and unique. Identifiers are used to identify named elements, such as variables, struct fields, and functions. Note that identifiers cannot begin with a number. The following is a regular expression for identifiers:
		
		\begin{verbatim}
ID = "['a'-'z' 'A'-'Z']['a'-'z' 'A'-'Z' '0'-'9' '_']*"
		\end{verbatim}
	\subsection{Reserved Keywords}
		The following is a list of reserved Democritus keywords:
        \begin{verbatim}
if          else        for         return      int     
bool        void        true        false       string
struct      *void       function    let
        \end{verbatim}
	\subsection{Literals}
		Literals are used to represent various values or constants within the language.

        \subsubsection{Integer Literals}
            Integer literals are simply a sequence of ASCII digits, represented in decimal. 
	       \begin{verbatim}
INT = "['0'-'9']+"
	       \end{verbatim}

        \iffalse
        \subsubsection{Float Literals}
            Float literals represent floating point numbers, also expressed in decimal. They can use the dot (.) notation, exponential notation, or any combination of the two. 

            \begin{verbatim}
FLOAT = 
            \end{verbatim}
        \fi
        \subsubsection{Boolean Literals}
            Boolean literals represent the two possible values that boolean variables can take, \texttt{true} or \texttt{false}. These literals are represented in lowercase. 
            \begin{verbatim}
BOOLEAN = "true|false"
            \end{verbatim}

        \subsubsection{String Literals}
            String literals represent strings of characters, including escaped characters. String literals are automatically null-terminated. Strings are opened and closed with double quotations. A special OCaml \texttt{lexbuf} was used to parse string literals. 

            \begin{verbatim}
(Taken from http://realworldocaml.org/v1/en/html/parsing-with-ocamllex-and-menhir.html)

read_string buf = parse
  | '"'             { STRING (Buffer.contents buf) }
  | '\\' '/'        { Buffer.add_char buf '/'; read_string buf lexbuf }
  | '\\' '\\'       { Buffer.add_char buf '\\'; read_string buf lexbuf }
  | '\\' 'b'        { Buffer.add_char buf '\b'; read_string buf lexbuf }
  | '\\' 'f'        { Buffer.add_char buf '\012'; read_string buf lexbuf }
  | '\\' 'n'        { Buffer.add_char buf '\n'; read_string buf lexbuf }
  | '\\' 'r'        { Buffer.add_char buf '\r'; read_string buf lexbuf }
  | '\\' 't'        { Buffer.add_char buf '\t'; read_string buf lexbuf }
  | [^ '"' '\\']+   { Buffer.add_string buf (Lexing.lexeme lexbuf);
                        read_string buf lexbuf
                    }
            \end{verbatim}


	\subsection{All Democritus Tokens}
		The list of tokens used in Democritus are as follows:

		\begin{verbatim}
| "//"     		{ comment lexbuf }             	
| "/*"     		{ multicomment lexbuf }          
| '('      		{ LPAREN }
| ')'      		{ RPAREN }
| '{'      		{ LBRACE }
| '}'      		{ RBRACE }
| ';'      		{ SEMI }
| ','      		{ COMMA }
| '+'      		{ PLUS }
| '-'      		{ MINUS }
| '*'      		{ STAR }
| '&'       { REF }
| '.'      		{ DOT }
| '/'      		{ DIVIDE }
| '='      		{ ASSIGN }
| "=="     		{ EQ }
| "!="     		{ NEQ }
| '<'      		{ LT }
| "<="     		{ LEQ }
| ">"      		{ GT }
| ">="     		{ GEQ }
| "&&"     		{ AND }
| "||"     		{ OR }
| "!"      		{ NOT }
| "if"     		{ IF }
| "else"   		{ ELSE }
| "for"    		{ FOR }
| "return" 		{ RETURN }
| "int"    		{ INT }
| "bool"   		{ BOOL }
| "void"   		{ VOID }
| "true"   		{ TRUE }
| "string"		 { STRTYPE }
| "struct" 		{ STRUCT }
| "*void"  		{ VOIDSTAR }
| "false"  		{ FALSE }
| "function"  { FUNCTION }
| "let"      	{ LET }
| ['0'-'9']+ as lxm { LITERAL(int_of_string lxm) }
| ['a'-'z' 'A'-'Z']['a'-'z' 'A'-'Z' '0'-'9' '_']* as lxm { ID(lxm) }
| '"'    { read_string (Buffer.create 17) lexbuf } (* refer to read_string above *)
| eof { EOF }
\end{verbatim}
		

	\subsection{Punctuation}
		\subsubsection{Semicolon}
			The semicolon `;' is required to terminate any statement in Democritus. 
			
			\begin{verbatim}
statement SEMI
			\end{verbatim}
			
		\subsubsection{Curly Brackets}
			In order to keep the language free-format, curly braces are used to delineate separate and nested blocks. These braces are required even for single-statement conditional and iteration loops.
			
			\begin{verbatim}
LBRACE statements RBRACE
			\end{verbatim}
			
		\subsubsection{Parentheses}
			To assert precedence, expressions may be encapsulated within parentheses to guarantee order of operations. 
			
			\begin{verbatim}
LPAREN expression RPAREN
			\end{verbatim}
			
		\subsection{Comments}
			Comments may either be single-line, intialized with two backslashes, or multi-line, enclosed by \verb|\* and *\|.

            \begin{verbatim}
COMMENT = ("// [^'\n']* \n") | ("/*" [^ "*/"]* "*/")
            \end{verbatim}

