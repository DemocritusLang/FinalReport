\section{Functions}
    \subsection{Overview and Grammar}
        Functions can be defined in Democritus to return one or no data type.  Functions are evaluated via eager (applicative-order) evaluation and the function implementation must directly follow the function header. The grammar for function declarations is as follows: 

        \begin{verbatim}
fdecl:
   FUNCTION ID LPAREN formals_opt RPAREN typ LBRACE vdecl_list stmt_list RBRACE 

formals_opt:
    /* nothing */
  | formal_list  

formal_list:
    ID typ                
  | formal_list COMMA ID typ 
        \end{verbatim}

        \noindent All functions require \texttt{return} statements at the end, and must return an expression of the same type as the function. \texttt{void} functions may simply terminate with an empty \texttt{return} statement. 
    

        \begin{lstlisting}
    }
function function_name([formal_arg type, ... ]) type_r {
    
    [function implementation]
    return [variable of type type_r]
}
        \end{lstlisting}


    \subsection{Calling and Recursion}

        Functions may be recursive and call themselves:

        \begin{lstlisting}
function recursive_func(i int) void {

    if (i < 0) {
        return;
    } else {
        print(“hi”);
        recursive_func(i-1);    // Call ourselves again.
    }
}
        \end{lstlisting}


        \noindent Functions may be called within other functions:
        \begin{lstlisting}

function main() void{
    recursive_func(3);
    return;                     // Return nothing for void.
}
        \end{lstlisting}

