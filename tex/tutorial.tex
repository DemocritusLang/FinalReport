\chapter{Language Tutorial}

Democritus is a statically-typed, imperative language with standard methods for conditional blocks, iteration, variable assignment, and expression evaluation. In this chapter, we will cover environment configuration as well as utilizing bothDemocritus' basic and advanced features.

\section{Setup and Installation}

	To set up the Democritus compiler, OCaml and LLVM must be installed. Testing and development was done in both native Ubuntu 15.04 and Ubuntu 14.04 running on a virtual machine.

	\begin{lstlisting}
sudo apt-get install m4 clang-3.7 clang clang-3.7-doc libclang-common-3.7-dev libclang-3.7-dev libclang1-3.7 libclang1-3.7-dbg libllvm-3.7-ocaml-dev libllvm3.7 libllvm3.7-dbg lldb-3.7 llvm-3.7 llvm-3.7-dev llvm-3.7-doc llvm-3.7-examples llvm-3.7-runtime clang-modernize-3.7 clang-format-3.7 python-clang-3.7 lldb-3.7-dev liblldb-3.7-dbg opam llvm-runtime
	\end{lstlisting}

	\medskip \noindent For Ubuntu 15.04, we need the matching LLVM 3.6 OCaml Library.

	\begin{lstlisting}
sudo apt-get install -y ocaml m4 llvm opam
opam init
opam install llvm.3.6 ocamlfind
eval `opam config env`
\end{lstlisting}

	\medskip \noindent For Ubuntu 14.04:

	\begin{lstlisting}
sudo apt-get install m4 llvm software-properties-common

sudo add-apt-repository --yes ppa:avsm/ppa
sudo apt-get update -qq
sudo apt-get install -y opam
opam init

eval `opam config env`

opam install llvm.3.4 ocamlfind
	\end{lstlisting}

	\medskip \noindent After setting up the environment, clone the git repository into your desired installation directory:

	\begin{lstlisting}
git clone https://github.com/DemocritusLang/Democritus.git
	\end{lstlisting}


\section{Compiling Your Code}
	
	To build the compiler, \texttt{cd} into the Democritus repository, and run \texttt{make}.

	\medskip \noindent
	If building fails, try running \texttt{eval `opam config env`}, which should update your local environment use OPAM packages and compilers. It's recommended to add the above command to your shell's configuration file if you plan on developing with Democritus.

	\medskip \noindent
	To compile code, simply run
	\begin{lstlisting}
./Democritus < filename.dem > outfile.lli
	\end{lstlisting}

	\noindent To run compiled code, call \texttt{lli} on the output:
	\begin{lstlisting}
lli outfile.lli
	\end{lstlisting}

\section{Writing Code}

	Code can be written in any text file, but Democritus source files should have the \texttt{.dem} extension by convention. Democritus programs consist of global function, struct, and variable declarations. Only the code inside main() will be exectued at runtime. At this time, linking is not included in the Democritus compiler; all code should be written and compiled from a single \texttt{.dem} source file.

\section{Getting Started}

	\subsection{Declarations}
	Functions are declared with the \texttt{<function func\_name(a type, b type) return\_type>} syntax. Variables are declared with the \texttt{<let var\_name var\_type;>} syntax. Statements are terminated with the semicolon \texttt{;}. Note that all variable declarations must happen before statements (including assignments) in any given function.

	\begin{lstlisting}
function triangle_area(base int, height int) int{
	return base*height/2;
}
	\end{lstlisting}

	\subsection{Types}

		\subsubsection{Primitives}
		Primitive types in Democritus include booleans and integers. The void type is also used for functions.

		\subsubsection{Strings}
		Strings are built-in to Democritus. String literals are added to global static memory at runtime, and string variables point to the literals. These literals are automatically null-terminated.

		\begin{lstlisting}
function main() int{
	let s string;
	let foo int;
	let bar bool;

	bar = true;
	s = "Hello, World!"
	foo = 55;
	bar = false;

	return 0;
}
		\end{lstlisting}


	 	\subsubsection{Structs}
	 	Structs are declared at the global level with the \verb|<struct struct_type { named fields }>| syntax. Struct declarations may also be nested.

	 	\begin{lstlisting}
struct Person{
	let name string;
	let age int;
	let info struct Info;
}

struct Info{
	let education string;
	let salary int;
}


function main() int{
	let p struct Person;

	p.name = "Joe";
	p.age = 30;
	p.info.education = "Bachelor's";
	p.info.salary = 99999;

	print(p.name);
	print(" earns: ");
	print_int(p.info.salary);

	return 0;
}
	 	\end{lstlisting}


	\iffalse
	\subsection{Modifiers}
		Information about atomic and pointers here.
	\fi

 	\subsection{Operators}
		
		Democritus includes the `standard' set of operators, defined as follows:

		\subsubsection{Binary Operators:}
		\begin{itemize}
			\item artithmetic: \texttt{+, -, *, /, \%}
			\item logical:\verb^ ==, !=, <, <=, >, >=, && (and), || (or) ^ 
		\end{itemize}

		\subsubsection{Unary Operators:}
		\begin{itemize}
			\item artihmetic: \texttt{-}
			\item logical: \texttt{! (not)}
			\item addressing: \texttt{\& (reference), * (reference)}
		\end{itemize}

		\noindent
		Logical expressions return a boolean value.

		\medskip \noindent
		The expressions on each side of a binary operation must be of the same type. The \verb^ &&, ||,^ and \verb^! ^ operators must be called on boolean expressions.

		\medskip \noindent
		References can only be called on addressable fields (such as variables, or struct fields). Dereferences can only be called on pointer types.


\section{Control Flow}
	As an imperative language, Democritus executes statements sequentially from the top of any given function to the bottom. Branching and iteration is done similarly to many other imperative languages.

	\subsection{Conditional Branching}
		Conditional branching is done with:

		\begin{lstlisting}
if(boolean expression) 
{ 
	/* do something here */
}

else
{
	/* do alternative here */
}
		\end{lstlisting}
	
		\medskip \noindent
		Here is an example of conditional branching in Democritus:

		\begin{lstlisting}
struct Person{
	let education string;
	let name string;
	let age int;
	let working bool;
}


function main() int{
	let p Struct person;
	p.name = "Joe"
	p.education = "Bachelor's";
	p.age = 25;
	p.working = false;

	if(p.working){
		print(p.name);
		print(" works.\n");
	}else{
		print(p.name);
		print(" is looking for work.\n");
	}

	return 0;
}
		\end{lstlisting}

		\medskip \noindent
		This program prints ``Joe is looking for work.''

	\subsection{Loops and Iteration}
		Iteration can be done either via a \texttt{for} loop. A \texttt{for(e1; e2; e3)} loop may take three expressions; \texttt{e1} is called prior to entering the loop, \texttt{e2} is a boolean conditional statement for the loop, and \texttt{e3} is called after each iteration. Both \texttt{e1} and \texttt{e3} are optional; omitting both converts the \texttt{for} loop into the conditional \texttt{while} used in other languages.

		\begin{lstlisting}
function main() int{
	let i int;
	for(i = 0; i<42; i=i+1){
		i = i+1;
	}

	for(;false;){
		// This block will never be reached
	}

	for(true){
		print_int(i);
	}
	return 0;
}
		\end{lstlisting}

		\medskip \noindent
		This program will print 42 forever.



\section{Multithreading and Atomicity}
	Democritus supports threading with the \texttt{thread()} function call, which then calls the underlying \texttt{pthread} function in C. Any defined function can be called with multiple threads. The calling syntax is as follows:
	\begin{lstlisting}
thread("functionname", <comma separated args>, #threads);
	\end{lstlisting}

	\noindent
	Multithreaded functions must take a \texttt{*void} type as input and return a starvoid to conform with C's calling convention. An example of a multithreaded program:

	\begin{lstlisting}
function multiprint(noop *void) *void{
	let x starvoid;
	print("Hello, World!\n");
	return x;
}

function main() int{
	thread("multiprint", 0, 6); /* "Hello, World!" will be printed six times. */
	return 0;
}

	\end{lstlisting}



\section{Miscellaneous}
	Besides threading, a couple of other functions from C have been bound to Democritus.

	\subsection{Malloc}
		\texttt{malloc(size)} may be called, returning a pointer to a newly heap-allocated block of $size$ bytes. These pointers may be bound to strings, which themselves are pointers to string literals. An example utilizing malloc will be included with file I/O.

	\subsection{Pointers}

		Democritus features a pointer type, a numerical value which points to an address in memory. 

		\medskip\noindent A pointer type is defined \verb|<* type>|, and as such they can be declared as follows: \verb| let a *int;|	
		
		\smallskip\noindent
		Dereferencing is performed with \texttt{*} (e.g. \verb|*a|), and may only be performed on pointer types. Referencing, which returns the memory address of a variable or addressable location in memory, is performed with \verb|&a|.

		\medskip\noindent
		Since malloc returns a \texttt{*void} type, it must be casted in order to be used or accessed as another type. To cast an a \texttt{malloc}'d pointer:
		\begin{lstlisting}
let i *int;
i = cast malloc(num_bytes) to *int;
		\end{lstlisting}

		\noindent
		With pointers and heap-allocated memory from \texttt{malloc()}, we are able to build up basic data structures. 

	\subsection{File I/O}
		Files may be opened with \texttt{open()}. This call returns an integer, which may be then bound as a file descriptor. C functions such as \texttt{write()}, \texttt{read()}, or \texttt{lseek()} may then be called on the file descriptor. 

		\begin{itemize}
			\item \texttt{open(filename, fd, fd2)}: opens a file. \texttt{filename} is a string referring to a file to be opened. \texttt{Fd} and \texttt{fd2} are file descriptors used for \texttt{open}.
			\item \texttt{write(fd, text, length)}: writes to a file. \texttt{fd} refers to the file descriptor of an open file. \texttt{text} is a string representing text to be written. \texttt{length} is an integer specifying the number of bytes to be written.
			\item \texttt{read(fd, buf, length)}: reads from a file. \texttt{fd} refers to the file descriptor of an open file. \texttt{buf} is a pointer to malloc'd or allocated space. \texttt{length} is an integer representing amount of data to be read. Buffers should be \texttt{malloc}'d before reading. 
			\item \texttt{lseek(fd, offset, whence)}: sets a file descriptors cursor position. \texttt{fd} is the file descriptor to an open file. \texttt{offset} is an integer describing how many bytes the cursor should be offset by, and \texttt{whence} is an integer describing how \texttt{offset} should be applied: as an absolute location, relative location to the current cursor, or relative location to the end of the file.
		\end{itemize}

		\noindent
		For more detailed information on these calls, run \texttt{man function\_name}.

		\begin{lstlisting}
function main() int{

  let fd int;
  let malloced string;
  
  fd = open("tests/HELLOOOOOO.txt", 66, 384); 	/* Open this file */
  write(fd, "hellooo!\n", 10); 									/* Write these 10 bytes */
  
  malloced = malloc(10); 			/* Allocate space for the data and null terminator */
  lseek(fd, 0, 0);						/* Jump to the front of the file */
  read(fd, malloced, 10);			/* Read the data we just wrote into the buffer */
  
  print(malloced);						/* Prints "hellooo!\n" */
  return 0;
}
		\end{lstlisting}

	\subsection{Sockets API}
	Democritus provides support for networking functionality through use of the C sockets API. A bound C function, \texttt{request\_from\_server}, allows a user to retrieve the contents of a webpage, written to a file, as follows:
	
			\begin{lstlisting}
function main() int
{
    let fd int;  //the file descriptor
    request_from_server("www.xkcd.com/index.html"); //write the content of the page to "index.html"
    exec_prog("/bin/cat", "cat", "tests/index.html"); //cat the file to stdout, used for testing 
    return 0;
}
		\end{lstlisting}
