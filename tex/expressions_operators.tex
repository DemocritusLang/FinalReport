\section{Expressions and Operators}
  In the grammars shown in this section and all subsequent sections, all terminals are expressed in uppercase and all nonterminals are kept lowercase. 

	\subsection{Expressions}
    Expressions may be any of the following:
    
    \subsubsection{Literal}
      A literal of any type, as detailed in the lexical conventions section.
    \subsubsection{Identifier}
      An identifier for a variable.
    \subsubsection{Binary Operation}
      A binary operation between an expression and another expression.
    \subsubsection{Unary Operation}
      A unary operation acting on the expression appearing on the immediate right of the operator.
    \subsubsection{Struct Access}
      An expression of a \texttt{struct} type accessing an identifier field with the \texttt{dot (.)} operator.
    \subsubsection{Struct Assignment}
      An expression of a \texttt{struct} type assigning a value to one of its fields (accessed with the \texttt{dot (.)} operator) using the \texttt{=} operator.
    \subsubsection{Function Call}
      A call to a function along with its formal arguments.
    \subsubsection{Variable Assignment}
      An identifier being assigned a value with the \texttt{=} operator.
    \subsubsection{Parenthisization}
      Another expression nested within parentheses.

    \bigskip\noindent The grammar for expressions is as follows:
    \begin{verbatim}
expr:
    LITERAL       
  | FLOATLITERAL     
  | TRUE          
  | FALSE            
  | ID            
  | STRING      
  | expr PLUS   expr              (* expr TERMINAL expr are binary operations *)
  | expr MINUS  expr 
  | expr STAR   expr 
  | expr DIVIDE expr 
  | expr MOD    expr 
  | expr EQ     expr 
  | expr NEQ    expr 
  | expr LT     expr 
  | expr LEQ    expr 
  | expr GT     expr 
  | expr GEQ    expr 
  | expr AND    expr 
  | expr OR     expr
  | expr DOT    ID                (* struct acccess     *)
  | expr DOT    ID ASSIGN expr    (* struct assign      *)
  | MINUS expr  %prec NEG         (* unary arith negate *)
  | STAR expr   %prec DEREF       (* unary deref        *)
  | REF expr                      (* unary ref          *)
  | NOT expr                      (* unary log negate   *)
  | ID ASSIGN expr  
  | ID LPAREN actuals_opt RPAREN  (* function call      *)
  | LPAREN expr RPAREN            (* paren'd expr       *)
    \end{verbatim}


    \subsection{Binary and Unary Operations}
      A binary operation operates on the two expressions on the left and right side of the operator. Binary operations may be:
    \begin{itemize}
      \item an addition, subtraction, mult., division, or modulo on two arithmetic expressions (\texttt{+,-,*,/,\%}).\\ Modulo only works on integer types.
      \item  equality or inequality expression between boolean expressions (\verb^==,!=,<,<=,>,>=,&&,||^)
    \end{itemize}

    \noindent
    A unary operation operates on the expression on the operator's right side:
    \begin{itemize}
      \item a negation of an arithmetic expression (\texttt{-})
      \item a dereference of a pointer type (\texttt{*})
      \item an address reference of a variable or field within a struct (\verb|&|)
      \item a negation of a boolean expression (\texttt{!})
    \end{itemize}


      \iffalse
  		\medskip \noindent 
      Array assignment is done with brackets. Note that the size of the array must be specified in the declaration. 
  		\fi
		
		
	\subsection{Arithmetic Operations}
		Democritus supports all the arithmetic operations standard to most general-purpose languages, documented below. Automatic casting has not been included in the language, and the compiler will throw an error in the case that arithmetic operations are performed between the same types of expressions.  


		
		\subsubsection{Addition and Subtraction}
			Addition works with the \texttt{+} character, behaving as expected. Subtraction is called with \texttt{-}.
						
		\subsubsection{Multiplication and Division} 
		  Multiplication is called with \texttt{*}, and division with \texttt{/}. Division between integers discards the fractional part of the division. 	


    \subsubsection{Modulo}
      The remainder of an integer division operation can be computed via the modulo \texttt{\%} operator.
	    \iffalse
		\subsubsection{Bit Shifting}
			Integers can be bit-shifted with \verb|>>| and \verb|<<|.
	  \fi

	\subsection{Boolean Expressions}
		Democritus features all standard logical operators, utilizing \texttt{!} for negation, and \verb|&&| and \verb^||^ for \texttt{and} and \texttt{or}, respectively. Each expression will return a boolean value of true or false.

    \subsubsection{Equality}
            Equality is tested with the \texttt{==} operator. Inequality is tested with \texttt{!=}. Equality may be tested on both boolean and arithmetic expressions. 
			
		\subsubsection{Negation}
			Negation is done with \texttt{!}, a unary operation for boolean expressions. 

    \subsubsection{Comparison}
      Democritus also features the \verb|<|, \verb|<=|, \verb|>|, and \verb|>=| operators. These represent less than, less than or equal to, greater than, and greater than or equal to, respectively. These operators are called on arithmetic expressions and return a boolean value. 
            

      \subsubsection{Chained Expressions}
            Boolean expressions can be chained with \verb|&&| and \verb!||!, representing \texttt{and} and \texttt{or}. These operators have lower precedence than any of the other boolean operators described above. The \texttt{and} operator has a higher precedence than \texttt{or}.

  \subsection{Parentheses}
    Parentheses are used to group expressions together, since they have the highest order of precedence. Using parentheses will ensure that whatever is encapsulated within will be evaluated first.

  \subsection{Function Calls}
    Function calls are treated as expressions with a type equal to their return type. As an applicative-order language, Democritus evaluates function arguments first before passing them to the function. The grammar for function calls is as follows:

    \begin{verbatim}
expr:
  .
  .
  .
  | ID LPAREN actuals_opt RPAREN { Call($1, $3) }    (* Function call *)

actuals_opt:
  /* nothing *?   { [] }
  |  actuals_list { List.rev $1 }

actuals_list:
     expr        { [$1] }
  |  actuals_list COMMA expr { $3 :: $1 }
    \end{verbatim}


  \subsection{Pointers and References}
		Referencing and dereferencing operations are used to manage memory and addressing in Democritus. The unary operator \texttt{\&} gives a variable or struct field's address in memory, and the operator \texttt{*} dereferences a pointer type.
	
  \iffalse
	\subsection{Array access}
		Array access is done with \texttt{[\textit{i}]} where \textit{i} is the index being accessed. 
  \fi

    \subsection{Operator Precedence and Associativity}
        \begin{tabular}{ | l | l | l | l | }\hline
	Precedence  & Operator      & Description & Associativity \\ \hline
	1 & \texttt{()} & Parenthesis        & Left-to-right \\ \hline
    2 & \texttt{()} & Function call      & Left-to-right \\
      \iffalse & \verb|{}| & Array creation  s   &               \\ 
      & \texttt{[]} & Array subscript    &               \\ \hline \fi
    3 & \texttt{*}  & Dereference        & Right-to-left \\ 
      & \texttt{\&} & Address-of         &               \\
      & \texttt{!}  & Negation           &               \\
      & \texttt{-}  & Unary minus        &               \\ \hline
    4 & \texttt{*}  & Multiplication     & Left-to-right \\ 
      & \texttt{/}  & Division           &               \\ 
      & \texttt{\%} & Modulo             &               \\ \hline 
    5 & \texttt{+}  & Addition           & Left-to-right \\
      & \texttt{-}  & Subtraction        &               \\ \hline
  \iffalse  6 & \verb|>>| & Bitwise shift shift right & Left-to-right \\
      & \verb|<<| & Bitwise shift left & \\ \hline \fi
    6 & \verb|>>| = & For relational $>$ and $\geq$ respectively & Left-to-right \\
      & \verb|<<=| & For relational $<$ and $\leq$ respectively & \\ \hline
    7 & \texttt{==} \texttt{!=} & For relational $=$ and $\neq$ respectively & Left-to-right \\\hline
    8 & \texttt{\&\&} & Logical \texttt{and} & Left-to-right \\ \hline
    9 & \verb!||! &  Logical \texttt{or} & Left-to-right \\ \hline
    10 & \texttt{=} & Assignment & Right-to-left \\ \hline




\end{tabular}


