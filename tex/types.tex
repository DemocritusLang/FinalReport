\section{Data types}
    \subsection{Primitive Types}
        \subsubsection{int}
            A standard 32-bit two's-complement signed integer. It can take any value in the inclusive range (-2147483648, 2147483647).
        \iffalse\subsubsection{float}
            A 64-bit floating precision number, represnted in the IEEE 754 format.            
        \subsubsection{char}
            An 8-bit ASCII character. We include the extended ASCII set, so we use all 256 possible values.\fi
        \subsubsection{boolean}
            A 1-bit true or false value.
        \iffalse\subsubsection{pointer}
            A 64-bit pointer that holds the value to a location in memory; pointers may be passed and dereferenced.\fi
    \subsection{Complex Types}
         \iffalse\subsubsection{Array}
            A fixed-size array, allocated on the stack and containing other primitive types. The size must be defined at declaration. An array object can be accessed by standard bracket notation, i.e. \texttt{list1[0]}.\fi
        \subsubsection{string}
            An immutable array of characters, implemented as a native data type in Democritus.
       \subsubsection{struct}
            A struct is a simple user-defined data structure that holds various primitives, similar to the ones found in C. 

